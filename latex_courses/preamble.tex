% ==============================
% Preambolo standard Sapienza Open Notes
% ==============================
\usepackage[utf8]{inputenc}
\usepackage[T1]{fontenc}
\usepackage[italian]{babel}

% Font elegante e leggibile
\usepackage{lmodern} 
\usepackage{microtype} % Migliora la spaziatura

% Matematica
\usepackage{amsmath, amssymb, amsthm, mathtools}

% Grafici e immagini
\usepackage{graphicx}
\usepackage{float} % Per controllare la posizione delle figure

% Collegamenti ipertestuali
\usepackage[hidelinks]{hyperref}

% Liste personalizzate
\usepackage{enumitem}

% Layout e geometria
\usepackage{geometry}
\geometry{a4paper, margin=2.5cm} 

% Stile dei paragrafi
\setlength{\parskip}{0.8em}
\setlength{\parindent}{0pt}

% Colori accento in stile Sapienza
\usepackage{xcolor}


\definecolor{sapienzaRed}{HTML}{822433}
\definecolor{sapienzaLightRed}{HTML}{F0C0C0}

% Comandi utili
\newcommand{\R}{\mathbb{R}}
\newcommand{\N}{\mathbb{N}}
\newcommand{\Z}{\mathbb{Z}}
\newcommand{\Q}{\mathbb{Q}}

% Numerazione teoremi
\usepackage{amsthm}
\newtheorem{theorem}{Teorema}[section]
\newtheorem{definition}{Definizione}[section]
\newtheorem{example}{Esempio}[section]

% ==============================
% Miglioramenti estetici
% ==============================
% Titoli più eleganti
\usepackage{titlesec}
\titleformat{\section}
  {\normalfont\Large\bfseries\color{sapienzaRed}}
  {\thesection}{1em}{}

\titleformat{\subsection}
  {\normalfont\large\bfseries\color{sapienzaRed!80!black}}
  {\thesubsection}{1em}{}

% Stile box per esempi e definizioni
\usepackage{tcolorbox}
\tcbset{
  colback=sapienzaLightRed!20, 
  colframe=sapienzaRed, 
  boxrule=0.8pt, 
  arc=4pt, 
  left=2mm, right=2mm, top=1mm, bottom=1mm,
  fonttitle=\bfseries
}


% ==============================
% Messaggio di benvenuto e invito a collaborare
% ==============================
\newcommand{\projectintro}{
  \section*{Contribuisci anche tu}
  Questi appunti fanno parte del progetto \textbf{Sapienza Open Notes}, 
  una raccolta collaborativa di materiali didattici per gli studenti della Sapienza Università di Roma.  

  Se trovi errori, omissioni o hai idee per migliorare gli appunti, 
  puoi contribuire aggiungendo correzioni o nuove lezioni.  
  Ogni contributo è prezioso e aiuta a creare una risorsa aperta e condivisa per tutti!  

  \medskip
  \noindent
  \textit{Per collaborare: visita la repository GitHub ufficiale} 
  \href{https://github.com/filippodaminato/SapienzaOpenNotes}{SapienzaOpenNotes}  
}

% ==============================
% Messaggio introduttivo e invito alla collaborazione
% ==============================
