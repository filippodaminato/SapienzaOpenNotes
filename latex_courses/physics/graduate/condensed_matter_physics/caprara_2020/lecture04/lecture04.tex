\section{Lezione 4: Diffrazione e Struttura Cristallina}
\label{appendix:lesson04}

\subsection{Introduzione: Come "vedere" i cristalli?}
Nelle lezioni precedenti abbiamo introdotto il formalismo matematico per descrivere la struttura periodica dei cristalli, basato sui concetti di reticolo di Bravais e reticolo reciproco. Ora ci poniamo una domanda sperimentale: come possiamo verificare che un dato solido sia cristallino e, in caso affermativo, determinarne la struttura?

La risposta risiede nell'interazione della materia con la radiazione. Per "vedere" un oggetto, abbiamo bisogno di una sonda la cui lunghezza d'onda sia comparabile o inferiore alle dimensioni dell'oggetto stesso. Poiché la distanza tipica tra atomi in un solido è dell'ordine dell'Ångström ($1\text{Å} = 10^{-10} \text{m}$), abbiamo bisogno di una radiazione con una lunghezza d'onda simile. La luce visibile, con lunghezze d'onda di migliaia di Ångström, non è adatta. I \textbf{raggi X}, invece, hanno lunghezze d'onda proprio nell'intervallo giusto.

Storicamente, lo studio della diffrazione dei raggi X da parte dei cristalli è la tecnica principe per la determinazione della loro struttura. Esistono due formulazioni principali, apparentemente diverse ma fondamentalmente equivalenti, per descrivere questo fenomeno: la legge di Bragg e la condizione di von Laue.

\subsection{La Formulazione di Bragg (Approccio Fenomenologico)}
La formulazione di William Lawrence Bragg, sviluppata prima di quella di von Laue, è più intuitiva e si basa su due assunzioni audaci ma efficaci:
\begin{enumerate}
    \item Un cristallo può essere visto come una pila di \textbf{piani reticolari} paralleli ed equidistanti.
    \item Questi piani si comportano come \textbf{specchi semi-trasparenti} per i raggi X, riflettendo una piccola frazione della radiazione incidente.
\end{enumerate}
Consideriamo un fascio di raggi X monocromatici di lunghezza d'onda $\lambda$ che incide su una famiglia di piani reticolari separati da una distanza $d$. Sia $\theta$ l'angolo di incidenza, definito come l'angolo tra il raggio incidente e il piano stesso (non la normale al piano).


Affinché le onde riflesse da due piani adiacenti diano luogo a interferenza costruttiva, la differenza di cammino ottico tra di esse deve essere un multiplo intero della lunghezza d'onda. Come si evince dalla geometria, questa differenza di cammino è pari a $2d\sin\theta$. Questo porta alla celebre \textbf{legge di Bragg}:

\begin{equation}
    2d\sin\theta = n\lambda, \quad \text{con } n = 1, 2, 3, ...
\end{equation}
dove $n$ è detto "ordine di diffrazione". Se questa condizione è soddisfatta, si osserva un picco di intensità riflessa. Misurando gli angoli $\theta$ per cui si osservano i picchi, e conoscendo $\lambda$, si possono determinare le distanze interplanari $d$ del cristallo.

\subsection{La Formulazione di von Laue (Approccio Microscopico)}
La trattazione di Max von Laue è più rigorosa e si basa su un'unica assunzione fisica: lo scattering dei raggi X da parte degli atomi del cristallo è \textbf{elastico}. Questo significa che l'energia del fotone X non cambia durante l'interazione, e quindi la sua lunghezza d'onda rimane costante. In questo modello, i centri diffusori (scatteratori) sono i singoli atomi che compongono il cristallo.

Consideriamo un'onda piana incidente, descritta dal vettore d'onda $\underline{k}$, che viene diffusa da tutti gli atomi del cristallo. Noi osserviamo l'onda diffusa in una certa direzione, descritta dal vettore d'onda $\underline{k}'$. Poiché lo scattering è elastico, i moduli dei due vettori d'onda sono uguali: $|\underline{k}| = |\underline{k}'| = 2\pi/\lambda$.

La condizione per l'interferenza costruttiva tra le onde diffuse da due atomi separati da un generico vettore del reticolo di Bravais $\underline{R}$ è che la differenza di fase tra le due onde sia un multiplo intero di $2\pi$. Questa differenza di fase è data dal prodotto scalare del \textbf{vettore scattering} $\Delta\underline{k} = \underline{k}' - \underline{k}$ con il vettore $\underline{R}$.
La condizione deve valere per \textit{tutti} gli atomi del cristallo, e quindi per ogni vettore $\underline{R}$ del reticolo di Bravais. Si ottiene così la \textbf{condizione di von Laue}:

\begin{equation}
    (\underline{k}' - \underline{k}) \cdot \underline{R} = 2\pi \times \text{intero} \quad \forall \underline{R} \in \text{Reticolo di Bravais}
\end{equation}

Questa equazione è la definizione stessa del reticolo reciproco. Pertanto, la condizione di von Laue si riduce a una semplice ma potente affermazione: si osserva un picco di diffrazione se e solo se il vettore scattering è un vettore del reticolo reciproco.

\begin{equation}
    \Delta\underline{k} = \underline{k}' - \underline{k} = \underline{G}
\end{equation}

Questo risultato è di fondamentale importanza: un esperimento di diffrazione dei raggi X mappa direttamente il \textbf{reticolo reciproco} del cristallo. Misurando la posizione dei picchi di diffrazione, si ricostruisce il reticolo reciproco e da questo, tramite le relazioni che lo definiscono, si risale univocamente al reticolo di Bravais reale.

\subsection{Equivalenza tra le formulazioni e la Sfera di Ewald}
Le due formulazioni sono equivalenti. Partendo dalla condizione di Laue, $\underline{k}' = \underline{k} + \underline{G}$, ed elevandola al quadrato, otteniamo $|\underline{k}'|^2 = |\underline{k}|^2 + |\underline{G}|^2 + 2\underline{k} \cdot \underline{G}$. Sfruttando la condizione di scattering elastico $|\underline{k}'|^2 = |\underline{k}|^2$, si arriva a:

\begin{equation}
    2\underline{k} \cdot \underline{G} + G^2 = 0
\end{equation}
Questa è un'altra forma della condizione di Laue. Ricordando che a ogni vettore $\underline{G}$ è associata una famiglia di piani reticolari (hkl) ad esso perpendicolari, e che la distanza tra questi piani è $d_{hkl} = 2\pi/|\underline{G}|$, si può dimostrare che questa equazione è identica alla legge di Bragg.

Una costruzione geometrica molto utile per visualizzare la condizione di diffrazione è la \textbf{sfera di Ewald}.
\begin{enumerate}
    \item Nello spazio reciproco, si disegna il vettore $\underline{k}$ del raggio incidente, facendolo terminare nell'origine del reticolo reciproco (il punto $\underline{G}=0$).
    \item Si costruisce una sfera di raggio $k = |\underline{k}|$ centrata nell'origine del vettore $\underline{k}$. Questa è la sfera di Ewald.
\end{enumerate}
La condizione di diffrazione $\underline{k}' = \underline{k} + \underline{G}$ è soddisfatta se e solo se la sfera di Ewald interseca un altro punto $\underline{G}$ del reticolo reciproco. In tal caso, il vettore che congiunge il centro della sfera con il punto $\underline{G}$ intersecato è il vettore d'onda diffratto $\underline{k}'$.

\subsection{Fattore di Struttura: Cristalli con Base}
Finora abbiamo considerato cristalli con un solo atomo per cella primitiva. Se il cristallo ha una \textbf{base} di più atomi, dobbiamo considerare l'interferenza tra le onde diffuse dai diversi atomi all'interno della stessa cella. L'ampiezza totale dell'onda diffusa in una direzione corrispondente a un vettore del reticolo reciproco $\underline{G}$ è data dal \textbf{fattore di struttura} $S(\underline{G})$.
Se la base è composta da $m$ atomi, con il $j$-esimo atomo in posizione $\underline{d}_j$ all'interno della cella, il fattore di struttura è:

\begin{equation}
    S(\underline{G}) = \sum_{j=1}^{m} f_j(\underline{G}) e^{i\underline{G}\cdot\underline{d}_j}
\end{equation}
L'intensità del picco di diffrazione è proporzionale a $|S(\underline{G})|^2$. Il termine $f_j(\underline{G})$ è il \textbf{fattore di forma atomico}, che descrive la capacità di scattering del singolo atomo $j$. Esso dipende dalla distribuzione spaziale degli elettroni dell'atomo ed è, in sostanza, la trasformata di Fourier della densità elettronica atomica.

\subsubsection*{Estinzioni Sistematiche: Il caso BCC e FCC}
Il fattore di struttura può essere zero per alcuni vettori $\underline{G}$ del reticolo reciproco, anche se la condizione di Laue sarebbe soddisfatta. Questo fenomeno, detto \textbf{estinzione sistematica}, fornisce informazioni cruciali sulla simmetria e sulla disposizione degli atomi.

\paragraph{Reticolo BCC:} Possiamo descriverlo come un reticolo cubico semplice (SC) con una base di due atomi identici, uno in $\underline{d}_1 = (0,0,0)$ e uno in $\underline{d}_2 = \frac{a}{2}(1,1,1)$. Il fattore di struttura è $S(hkl) = f[1 + e^{i\pi(h+k+l)}]$.
\begin{itemize}
    \item Se la somma degli indici di Miller $h+k+l$ è \textbf{pari}, $S(hkl) = 2f$. Il picco è visibile.
    \item Se $h+k+l$ è \textbf{dispari}, $S(hkl) = 0$. Il picco è \textbf{estinto}.
\end{itemize}
Ad esempio, nel reticolo reciproco di un BCC (che è un FCC), i picchi (100) e (300) sono assenti, mentre il picco (200) è presente.

\paragraph{Reticolo FCC:} Lo descriviamo come un reticolo SC con una base di 4 atomi identici in $(0,0,0)$, $\frac{a}{2}(1,1,0)$, $\frac{a}{2}(1,0,1)$, $\frac{a}{2}(0,1,1)$. Il fattore di struttura è $S(hkl) = f[1 + e^{i\pi(h+k)} + e^{i\pi(h+l)} + e^{i\pi(k+l)}]$.
\begin{itemize}
    \item Se gli indici $(h,k,l)$ sono \textbf{tutti pari o tutti dispari} (es. (111), (200), (220)), $S(hkl) = 4f$. Il picco è visibile.
    \item Se gli indici $(h,k,l)$ sono \textbf{misti} (alcuni pari e alcuni dispari, es. (100), (210)), $S(hkl) = 0$. Il picco è \textbf{estinto}.
\end{itemize}
L'analisi delle estinzioni sistematiche è quindi uno strumento fondamentale per distinguere, ad esempio, tra un reticolo SC, BCC e FCC.

\subsection{Effetto della Temperatura e delle Imperfezioni}
In un cristallo ideale e a temperatura zero, i picchi di diffrazione sarebbero infinitamente stretti (funzioni delta di Dirac). In un cristallo reale, ci sono sempre delle deviazioni dalla perfetta periodicità, dovute alle vibrazioni termiche degli atomi (fononi) o a difetti strutturali. Queste imperfezioni non distruggono i picchi di diffrazione (a meno che non siano così gravi da rendere il materiale amorfo), ma ne modificano due aspetti:
\begin{enumerate}
    \item \textbf{Intensità:} L'intensità dei picchi diminuisce all'aumentare della temperatura. Questo effetto è descritto dal fattore di Debye-Waller.
    \item \textbf{Larghezza:} I picchi non sono più infinitamente stretti ma acquisiscono una larghezza finita. La larghezza dei picchi è inversamente proporzionale alla dimensione del dominio di coerenza cristallina, cioè alla regione di spazio su cui si estende la periodicità perfetta.
\end{enumerate}