\section{Dinamica Reticolare e Oscillazioni Armoniche}
\label{sec:lecture8}
In questa lezione iniziamo lo studio della dinamica dei nuclei in un cristallo, un argomento trattato nel capitolo 22 del testo di Ashcroft \& Mermin. Abbandoniamo temporaneamente lo studio delle proprietà elettroniche per concentrarci prima sulla fisica del reticolo cristallino.

\subsection{Le Approssimazioni Fondamentali}

Per trattare la dinamica di un sistema così complesso come un cristallo, che contiene un numero enorme di nuclei ed elettroni in interazione, è necessario introdurre delle approssimazioni fisicamente motivate.

\subsubsection{1. L'Approssimazione di Born-Oppenheimer}
La prima e più importante approssimazione è quella di \textbf{Born-Oppenheimer}. Il suo contenuto fisico si basa sulla grande differenza di massa tra elettroni e nuclei (\(m_{\text{nucleo}} \gg m_e\)).

\begin{itemize}
    \item Poiché gli elettroni sono molto più leggeri, si muovono su una scala temporale molto più rapida rispetto ai nuclei.
    \item Possiamo quindi, in prima approssimazione, considerare i nuclei come "congelati" in una certa configurazione istantanea.
    \item Si risolve il problema quantistico per i soli elettroni, che si muovono nel potenziale statico generato dai nuclei fissi.
    \item Il risultato di questo calcolo è l'energia dello stato fondamentale degli elettroni, che dipende parametricamente dalle posizioni \(\lbrace \mathbf{R} \rbrace\) di tutti i nuclei.
    \item Questa energia elettronica, \(E_{\text{el}}(\lbrace \mathbf{R} \rbrace)\), agisce come un \textbf{potenziale efficace} per il moto dei nuclei.
\end{itemize}
Questa è la stessa approssimazione utilizzata in fisica molecolare; un solido può essere visto come una molecola di dimensioni macroscopiche.

\subsubsection{2. L'Approssimazione Armonica}
La seconda approssimazione cruciale è quella \textbf{armonica}. L'idea di base è che i nuclei non sono fermi, ma oscillano attorno alle loro posizioni di equilibrio reticolare.

\begin{itemize}
    \item La posizione istantanea \(\mathbf{r}(\mathbf{R})\) di un nucleo, la cui posizione di equilibrio è data dal vettore del reticolo di Bravais \(\mathbf{R}\), può essere scritta come:
    \[ \mathbf{r}(\mathbf{R}) = \mathbf{R} + \mathbf{u}(\mathbf{R}) \]
    dove \(\mathbf{u}(\mathbf{R})\) è il vettore \textbf{spostamento} dalla posizione di equilibrio.
    \item L'approssimazione armonica consiste nell'assumere che l'ampiezza di queste oscillazioni sia \textbf{molto piccola} rispetto alla distanza interatomica (la costante reticolare \(a\)): \(|\mathbf{u}(\mathbf{R})| \ll a\).
    \item Questa condizione è generalmente ben soddisfatta a basse temperature. Ad alte temperature, le oscillazioni diventano così ampie da rompere la struttura cristallina (fusione del cristallo), un fenomeno descritto empiricamente dal \textbf{criterio di Lindemann}.
\end{itemize}

\subsection{Energia Potenziale nell'Approssimazione Armonica}

Consideriamo l'energia potenziale totale \(U\) del cristallo, che dipende dalle posizioni istantanee di tutti i nuclei. Assumendo interazioni a due corpi (trascurando interazioni a 3 o più corpi), possiamo scriverla come una somma su tutte le coppie di atomi:
\[ U = \frac{1}{2} \sum_{\mathbf{R} \neq \mathbf{R'}} \Phi(\mathbf{r}(\mathbf{R}) - \mathbf{r}(\mathbf{R'})) = \frac{1}{2} \sum_{\mathbf{R} \neq \mathbf{R'}} \Phi(\mathbf{R} - \mathbf{R'} + \mathbf{u}(\mathbf{R}) - \mathbf{u}(\mathbf{R'})) \]
Il fattore \(\frac{1}{2}\) corregge il doppio conteggio.

Ora espandiamo in serie di Taylor questa energia potenziale per piccoli spostamenti \(\mathbf{u}\).

\paragraph{Termine di ordine zero (energia di equilibrio).}
È l'energia del cristallo con tutti i nuclei nelle loro posizioni di equilibrio (\(\mathbf{u}=0\) per ogni \(\mathbf{R}\)).
\[ U_0 = \frac{1}{2} \sum_{\mathbf{R} \neq \mathbf{R'}} \Phi(\mathbf{R} - \mathbf{R'}) \]
Questo termine è una costante e rappresenta l'energia di coesione del cristallo. Poiché \(\Phi\) dipende solo dalla differenza \(\mathbf{R}-\mathbf{R'}\), la doppia somma può essere riscritta come:
\[ U_0 = \frac{N}{2} \sum_{\mathbf{R} \neq 0} \Phi(\mathbf{R}) \]
dove \(N\) è il numero totale di atomi. Questo dimostra che l'energia è una quantità \textbf{estensiva}, proporzionale alla dimensione del sistema.

\paragraph{Termine del primo ordine.}
Lo sviluppo al primo ordine coinvolge il gradiente del potenziale.
\[ U_1 \propto \sum_{\mathbf{R}, \mathbf{R'}} (\nabla \Phi)_{\mathbf{u}=0} \cdot (\mathbf{u}(\mathbf{R}) - \mathbf{u}(\mathbf{R'})) \]
Questo termine è \textbf{nullo}. La condizione di equilibrio stabile impone che la forza netta su ogni atomo sia zero, il che significa che il gradiente dell'energia potenziale calcolato nelle posizioni di equilibrio deve annullarsi.

\paragraph{Termine del secondo ordine (termine armonico).}
Questo è il primo termine non banale e descrive l'energia immagazzinata in un sistema di oscillatori accoppiati.
\[ U_{\text{harm}} = \frac{1}{4} \sum_{\mathbf{R}, \mathbf{R'}} \sum_{\mu, \nu} (u_\mu(\mathbf{R}) - u_\mu(\mathbf{R'})) (u_\nu(\mathbf{R}) - u_\nu(\mathbf{R'})) \frac{\partial^2 \Phi}{\partial x_\mu \partial x_\nu} \bigg|_{\mathbf{u}=0} \]
dove \(\mu, \nu\) sono gli indici cartesiani (x, y, z). L'approssimazione armonica consiste nel troncare lo sviluppo a questo ordine.

L'Hamiltoniana del nostro sistema diventa quindi:
\[ \mathcal{H} = \sum_{\mathbf{R}} \frac{\mathbf{p}(\mathbf{R})^2}{2M} + U_0 + U_{\text{harm}} \]
dove il primo termine è l'energia cinetica di tutti i nuclei (assumendo un cristallo monoatomico con massa M).

\subsection{La Matrice Dinamica}

L'espressione per \(U_{\text{harm}}\) può essere riscritta in una forma più compatta:
\[ U_{\text{harm}} = \frac{1}{2} \sum_{\mathbf{R}, \mathbf{R'}} \sum_{\mu, \nu} u_\mu(\mathbf{R}) D_{\mu\nu}(\mathbf{R} - \mathbf{R'}) u_\nu(\mathbf{R'}) \]
dove abbiamo introdotto la \textbf{matrice dinamica} \(D_{\mu\nu}\). Questa matrice descrive le "costanti di accoppiamento" tra le oscillazioni dei diversi atomi. 

\subsection{Modello Semplificato: Catena Monoatomica 1D}

Per comprendere la fisica delle vibrazioni reticolari, partiamo dal modello più semplice possibile: una catena lineare di atomi identici di massa \(M\), separati da una distanza \(a\) all'equilibrio, con interazione tra primi vicini descritta da una costante elastica \(K\).

L'equazione del moto per lo spostamento \(u_n\) dell'n-esimo atomo è:
\[ M \ddot{u}_n = K(u_{n+1} + u_{n-1} - 2u_n) \]
Questa è un'equazione differenziale alle differenze finite. Cerchiamo soluzioni di tipo \textbf{onda piana}:
\[ u_n(t) = A e^{i(qna - \omega t)} \]
dove \(q\) è il vettore d'onda e \(\omega\) la frequenza angolare.

Sostituendo questa soluzione nell'equazione del moto, si ottiene la \textbf{relazione di dispersione}:
\[ \omega^2 = \frac{4K}{M} \sin^2\left(\frac{qa}{2}\right) \]
da cui:
\[ \omega(q) = \sqrt{\frac{4K}{M}} \left| \sin\left(\frac{qa}{2}\right) \right| \]

\subsection{Analisi della Relazione di Dispersione e Fononi}

La relazione di dispersione \(\omega(q)\) descrive come la frequenza di vibrazione dipenda dal vettore d'onda.
\begin{itemize}
    \item \textbf{Limite di grandi lunghezze d'onda (\(q \to 0\))}: Per \(q\) piccolo, \(\sin(qa/2) \approx qa/2\). La relazione diventa lineare:
    \[ \omega(q) \approx \left( a\sqrt{\frac{K}{M}} \right) q = v_s q \]
    Questa è la relazione per le onde sonore, dove \(v_s\) è la \textbf{velocità del suono}. In questo limite, il reticolo si comporta come un mezzo elastico continuo.
    \item \textbf{Confini della Prima Zona di Brillouin (\(q = \pm \pi/a\))}: A causa della periodicità del reticolo, i vettori d'onda fisicamente unici sono confinati nell'intervallo \([-\pi/a, \pi/a]\), la \textbf{prima zona di Brillouin}. Ai bordi della zona, la velocità di gruppo \(v_g = d\omega/dq\) si annulla. Questo corrisponde a onde stazionarie, in cui atomi adiacenti oscillano in opposizione di fase.
\end{itemize}
La quantizzazione di queste onde vibrazionali dà origine a quasiparticelle chiamate \textbf{fononi}, i quanti delle vibrazioni reticolari, che si comportano come bosoni.

