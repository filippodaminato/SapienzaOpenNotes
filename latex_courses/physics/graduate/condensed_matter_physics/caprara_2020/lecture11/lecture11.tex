\section{Dinamica Reticolare in 3D e Conteggio dei Modi Vibrazionali}

\subsection{Introduzione e Riepilogo del Caso Generale}
In questa lezione, generalizziamo i concetti di dinamica reticolare visti per i modelli unidimensionali (monoatomico e diatomico) al caso più realistico di un \textbf{cristallo tridimensionale} con una \textbf{base di \(p\) atomi} per cella primitiva. Affronteremo il problema del conteggio corretto dei modi vibrazionali e introdurremo le condizioni al contorno di Born-von Karman, che sono essenziali per quantizzare i vettori d'onda e garantire che il numero di gradi di libertà sia conservato.

Consideriamo un cristallo descritto da \(N\) celle primitive, ognuna contenente una base di \(p\) atomi.
\begin{itemize}
    \item Il numero totale di atomi nel cristallo è \(N \times p\).
    \item Poiché ogni atomo può muoversi in tre direzioni (x, y, z), il numero totale di gradi di libertà del sistema è \(3Np\).
\end{itemize}
Nell'approssimazione armonica, il nostro sistema è descritto da \(3Np\) equazioni differenziali accoppiate.

\paragraph{Soluzioni a Onda Piana e Vettori di Polarizzazione.}
Sfruttando la simmetria traslazionale del reticolo, cerchiamo soluzioni a onda piana per lo spostamento \(\mathbf{u}^{j}(\mathbf{R})\) del \(j\)-esimo atomo (con \(j=1, \dots, p\)) nella cella primitiva situata in \(\mathbf{R}\):
\[ \mathbf{u}^{j}(\mathbf{R}, t) = \boldsymbol{\epsilon}^{j} e^{i(\mathbf{q}\cdot\mathbf{R} - \omega t)} \]
\begin{itemize}
    \item \(\mathbf{q}\) è il vettore d'onda, che è lo stesso per tutti gli atomi.
    \item \(\omega\) è la frequenza di vibrazione.
    \item \(\boldsymbol{\epsilon}^{j}\) è un vettore tridimensionale chiamato \textbf{vettore di polarizzazione}. Descrive l'ampiezza e la direzione dell'oscillazione del \(j\)-esimo atomo della base. Poiché ci sono \(p\) atomi, abbiamo \(p\) di questi vettori per ogni modo vibrazionale.
\end{itemize}

\paragraph{Il Problema agli Autovalori.}
Sostituendo queste soluzioni nelle equazioni del moto si ottiene un sistema di \(3p \times 3p\) equazioni algebriche lineari. La condizione di esistenza di soluzioni non banali porta a un problema agli autovalori. Per ogni vettore d'onda \(\mathbf{q}\) nella prima zona di Brillouin, dobbiamo risolvere un'equazione secolare che fornisce \(3p\) soluzioni per \(\omega^2\).
\[ \omega^2_{s}(\mathbf{q}), \quad s=1, 2, \dots, 3p \]
Queste \(3p\) soluzioni definiscono le \textbf{branche di dispersione} dei fononi.

\subsection{Classificazione delle Branche: Modi Acustici e Ottici}
Le \(3p\) branche di dispersione non sono tutte uguali. Si dividono in due categorie con proprietà fisiche distinte:
\begin{itemize}
    \item \textbf{3 Branche Acustiche:} Queste tre branche (una per ogni dimensione spaziale) hanno la proprietà fondamentale che la loro frequenza tende a zero linearmente quando il vettore d'onda tende a zero:
    \[ \lim_{\mathbf{q} \to 0} \omega_{A}(\mathbf{q}) = v_s |\mathbf{q}| \to 0 \]
    Fisicamente, per \(\mathbf{q} \to 0\) (grandi lunghezze d'onda), tutti gli \(p\) atomi nella cella primitiva si muovono \textbf{in fase e con la stessa ampiezza}, come un'unità rigida. Questo corrisponde a un'onda sonora (acustica) che si propaga attraverso il cristallo. Delle tre branche acustiche, una corrisponde a oscillazioni \textbf{longitudinali} (la polarizzazione \(\boldsymbol{\epsilon}\) è parallela a \(\mathbf{q}\)) e due a oscillazioni \textbf{trasversali} (la polarizzazione \(\boldsymbol{\epsilon}\) è perpendicolare a \(\mathbf{q}\)).

    \item \textbf{\(3p-3\) Branche Ottiche:} Le rimanenti \(3p-3\) branche hanno una frequenza che tende a un valore \textbf{finito e non nullo} per \(\mathbf{q} \to 0\):
    \[ \lim_{\mathbf{q} \to 0} \omega_{O}(\mathbf{q}) = \omega_0 \neq 0 \]
    Fisicamente, in un modo ottico, gli atomi all'interno della stessa cella primitiva oscillano \textbf{l'uno contro l'altro} (in opposizione di fase). Il centro di massa della cella primitiva rimane fermo o compie oscillazioni molto piccole. Se il cristallo è ionico (es. NaCl), questo moto relativo tra cariche positive e negative crea un dipolo elettrico oscillante che può interagire fortemente con la luce (da cui il nome "ottico").
\end{itemize}

\subsection{Conteggio dei Modi e Condizioni al Contorno di Born-von Karman}
Abbiamo stabilito che ci sono \(3p\) branche di dispersione. Ma quanti valori distinti può assumere il vettore d'onda \(\mathbf{q}\)? La risposta dipende dalle dimensioni del cristallo e dalle condizioni al contorno imposte. Un cristallo infinito avrebbe un continuo di valori di \(\mathbf{q}\), ma un cristallo reale è finito.

\paragraph{Il Problema delle Superfici.}
Trattare esplicitamente le superfici di un cristallo è matematicamente proibitivo. Per aggirare questo problema, si utilizzano le \textbf{condizioni al contorno periodiche di Born-von Karman}.

\paragraph{Il Principio Fisico.}
L'idea è che per un cristallo macroscopico, le proprietà di volume (bulk) non dovrebbero dipendere da ciò che accade esattamente sulla superficie. Imponiamo quindi una condizione artificiale ma matematicamente conveniente: che il cristallo sia periodico su se stesso.
Consideriamo un cristallo a forma di parallelepipedo, con lati definiti dai vettori \(N_1\mathbf{a}_1, N_2\mathbf{a}_2, N_3\mathbf{a}_3\), dove \(\mathbf{a}_i\) sono i vettori primitivi e \(N_i\) sono interi molto grandi. Il numero totale di celle primitive è \(N = N_1N_2N_3\).
La condizione di Born-von Karman impone che lo spostamento di un atomo sia lo stesso in celle separate da una traslazione macroscopica del cristallo:
\[ \mathbf{u}(\mathbf{R}) = \mathbf{u}(\mathbf{R} + N_1\mathbf{a}_1) = \mathbf{u}(\mathbf{R} + N_2\mathbf{a}_2) = \mathbf{u}(\mathbf{R} + N_3\mathbf{a}_3) \]
Applicando questa condizione alla nostra soluzione a onda piana \(e^{i\mathbf{q}\cdot\mathbf{R}}\), otteniamo:
\[ e^{i\mathbf{q}\cdot\mathbf{R}} = e^{i\mathbf{q}\cdot(\mathbf{R} + N_j\mathbf{a}_j)} \implies e^{i\mathbf{q}\cdot N_j\mathbf{a}_j} = 1 \quad \text{per } j=1,2,3 \]
Questa condizione è soddisfatta se il prodotto scalare è un multiplo intero di \(2\pi\).

\paragraph{Quantizzazione del Vettore d'onda \(\mathbf{q}\).}
Esprimiamo \(\mathbf{q}\) nella base dei vettori del reticolo reciproco \(\mathbf{b}_i\):
\[ \mathbf{q} = x_1 \mathbf{b}_1 + x_2 \mathbf{b}_2 + x_3 \mathbf{b}_3 \]
Ricordando che \(\mathbf{a}_i \cdot \mathbf{b}_j = 2\pi\delta_{ij}\), la condizione \(e^{i\mathbf{q}\cdot N_j\mathbf{a}_j} = 1\) diventa:
\[ e^{i(x_j N_j 2\pi)} = 1 \implies x_j N_j = m_j \quad (\text{con } m_j \text{ intero}) \]
Quindi, i coefficienti \(x_j\) sono quantizzati:
\[ x_j = \frac{m_j}{N_j} \]
Il vettore d'onda \(\mathbf{q}\) può assumere solo un insieme discreto di valori:
\[ \mathbf{q} = \frac{m_1}{N_1}\mathbf{b}_1 + \frac{m_2}{N_2}\mathbf{b}_2 + \frac{m_3}{N_3}\mathbf{b}_3 \]
I valori fisicamente unici di \(\mathbf{q}\) sono quelli all'interno della prima zona di Brillouin. Questo corrisponde a scegliere gli interi \(m_j\) nell'intervallo \(0 \le m_j < N_j\).

\paragraph{Il Conteggio Finale.}
Il numero totale di valori permessi per il vettore d'onda \(\mathbf{q}\) è il prodotto del numero di scelte per ogni \(m_j\):
\[ \text{Numero di } \mathbf{q} \text{ permessi} = N_1 \times N_2 \times N_3 = N \]
Ci sono esattamente \(N\) vettori d'onda permessi, tanti quante sono le celle primitive nel cristallo.

\textbf{Conclusione Fondamentale:}
\begin{itemize}
    \item Abbiamo \(N\) valori distinti di \(\mathbf{q}\).
    \item Per ogni \(\mathbf{q}\), abbiamo \(3p\) branche di dispersione (modi vibrazionali).
    \item Il numero totale di modi vibrazionali indipendenti è \(N \times 3p = 3Np\).
\end{itemize}
Questo risultato è esattamente uguale al numero totale di gradi di libertà del nostro sistema. Le condizioni al contorno di Born-von Karman ci hanno permesso di contare correttamente tutti gli stati vibrazionali del cristallo, dimostrando la coerenza del nostro approccio.
