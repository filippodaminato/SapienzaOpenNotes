
\section{Introduzione: La Necessità di una Fusione}

Il punto di partenza del nostro corso è l'unione di due pilastri della fisica moderna: la \textbf{Meccanica Quantistica (MQ)} e la \textbf{Relatività Speciale (RS)}.

\begin{itemize}
    \item La \textbf{Meccanica Quantistica} fornisce una descrizione estremamente efficace del mondo microscopico. La sua validità è manifesta quando le quantità con le dimensioni di un'azione sono confrontabili con la costante di Planck, $\hbar$. Quando le scale di azione in un problema sono molto più grandi di $\hbar$, il suo contributo diventa trascurabile e si può ricorrere al limite semi-classico ($\hbar \to 0$).
    
    \item La \textbf{Relatività Speciale} si rivela fondamentale per descrivere fenomeni in cui le velocità in gioco, $v$, sono paragonabili alla velocità della luce, $c$. Per velocità molto minori ($v \ll c$), le leggi della meccanica Newtoniana sono sufficienti.
\end{itemize}

L'obiettivo di questo corso è esplorare la "fusione" di questi due ambiti. Si potrebbe pensare che sia sufficiente "mettere insieme" le regole già note. Tuttavia, vedremo che questo processo non è una semplice combinazione tecnica, ma richiede un vero e proprio \textbf{salto concettuale}. La fusione tra MQ e RS ci costringerà a riconsiderare e abbandonare alcuni dei concetti fondamentali a cui siamo abituati, primo tra tutti quello di funzione d'onda.

\subsection{Una Scala di Lunghezza Fondamentale: La Lunghezza d'Onda Compton}

Per iniziare, combiniamo le costanti fondamentali di queste due teorie, $\hbar$ e $c$, con una proprietà intrinseca di una particella microscopica come l'elettrone: la sua massa a riposo, $m_e$. Possiamo costruire una quantità che ha le dimensioni di una lunghezza:

$$
\lambda_C = \frac{\hbar}{m_e c}
$$

Questa è nota come \textbf{lunghezza d'onda Compton} dell'elettrone. Per calcolarne il valore, è utile riscriverla come:

$$
\lambda_C = \frac{\hbar c}{m_e c^2}
$$

Utilizzando i valori noti ($\hbar c \approx 197 \text{ MeV fm}$ e $m_e c^2 \approx 0.5 \text{ MeV}$):

$$
\lambda_C \approx \frac{197 \text{ MeV fm}}{0.5 \text{ MeV}} \approx 394 \text{ fm} \approx 4 \times 10^{-13} \text{ m}
$$

Questa è una scala di lunghezza drammaticamente piccola. Per dare un contesto, il raggio di Bohr (la dimensione tipica di un atomo di idrogeno) è dell'ordine di $10^{-10}$ m, quindi la lunghezza d'onda Compton è migliaia di volte più piccola.

\subsection{Il Principio di Indeterminazione nel Regime Relativistico}

Consideriamo ora le implicazioni del \textbf{Principio di Indeterminazione di Heisenberg}:

$$
\Delta q \cdot \Delta p \ge \hbar 
$$

In MQ non relativistica, se si localizza con precisione una particella ($\Delta q \to 0$), l'incertezza sulla sua quantità di moto diverge ($\Delta p \to \infty$). Questo fatto non ha mai rappresentato un problema concettuale insormontabile.

Le cose cambiano drasticamente quando introduciamo la relatività. Facciamo un'ipotesi audace (\textit{bold assumption}): supponiamo che esista una sorta di limite fisico alla localizzazione di una particella, una $(\Delta q)_{\min}$, e che questa sia proprio dell'ordine della lunghezza d'onda Compton:

$$
(\Delta q)_{\min} \sim \frac{\hbar}{m_e c}
$$

Cosa succede se proviamo a sondare una regione di spazio più piccola di questa, ovvero se $\Delta q < (\Delta q)_{\min}$? Dal principio di indeterminazione, segue che l'incertezza sul momento deve soddisfare:

$$
\Delta p > \frac{\hbar}{(\Delta q)_{\min}} = \frac{\hbar}{\hbar / (m_e c)} = m_e c
$$

Quindi, tentare di localizzare un elettrone in una regione più piccola della sua lunghezza d'onda Compton implica un'incertezza sulla sua quantità di moto \textbf{maggiore di $m_e c$}.

\subsubsection{Dall'Incertezza sul Momento all'Incertezza sull'Energia}

Vediamo come questa incertezza sul momento si traduce in un'incertezza sull'energia. Partiamo dalla relazione relativistica energia-momento:

$$
E = \sqrt{p^2 c^2 + m_e^2 c^4}
$$

Stimiamo la variazione di energia $\Delta E$ corrispondente a una variazione $\Delta p$:

$$
\Delta E \approx \left| \frac{dE}{dp} \right| \Delta p = \frac{p c^2}{E} \Delta p = v \Delta p
$$

Dato che abbiamo trovato $\Delta p > m_e c$, l'incertezza sull'energia diventa:

$$
\Delta E > v (m_e c)
$$

L'analisi dimensionale e le disuguaglianze sono coerenti con il fatto che l'incertezza sull'energia possa diventare significativamente grande, in particolare può superare una soglia critica:

$$
\Delta E \ge 2 m_e c^2
$$

\subsection{Creazione di Particelle e Crisi della Funzione d'Onda}

Il valore $E_0 = m_e c^2$ è l'energia a riposo di un elettrone. La soglia di $2 m_e c^2$ ha quindi un significato fisico profondo: è l'energia minima necessaria per \textbf{creare dal vuoto una coppia elettrone-positrone}. Un positrone è un'\textbf{antiparticella}: una particella con la stessa massa dell'elettrone ma carica elettrica opposta. La creazione di una coppia particella-antiparticella non viola la conservazione della carica elettrica.

Questo porta a una conseguenza sconvolgente: se tentiamo di misurare la posizione di un singolo elettrone con una precisione $\Delta q < \lambda_C$, l'energia che immettiamo nel sistema è così grande e indeterminata che potremmo, di fatto, creare altre particelle. Non siamo più nemmeno in grado di dire se nel nostro sistema ci sia un elettrone o tre (l'elettrone originale più una coppia elettrone-positrone).

Questo fenomeno segna il \textbf{crollo del concetto di funzione d'onda di Schrödinger, $\Psi(q)$}.
\begin{itemize}
    \item La funzione d'onda $\Psi(q)$ è, per definizione, la funzione d'onda di una \textbf{singola particella}.
    \item Se il numero di particelle nel sistema non è più conservato, che senso ha la "funzione d'onda dell'elettrone"? Di quale elettrone?.
    \item Inoltre, l'idea stessa di funzione d'onda si basa sulla possibilità, almeno in linea di principio, di conoscere l'argomento $q$ con precisione infinita. Se esiste una $\Delta q_{\min}$ al di sotto della quale non si può andare, il concetto stesso di $\Psi(q)$ diventa problematico.
\end{itemize}

Dobbiamo quindi abbandonare la funzione d'onda e cercare un formalismo nuovo, capace di descrivere sistemi con un numero variabile di particelle.

\subsection{Fluttuazioni Quantistiche e Particelle Virtuali}

Un altro modo per comprendere questo fenomeno è attraverso il \textbf{principio di indeterminazione energia-tempo}:

$$
\Delta E \cdot \Delta t \ge \hbar
$$

Questo principio implica che la conservazione dell'energia può essere violata, purché ciò avvenga per intervalli di tempo $\Delta t$ sufficientemente brevi. Su scale temporali molto piccole, un elettrone non è mai veramente "solo". La sua esistenza è caratterizzata da continue \textbf{fluttuazioni quantistiche}.


Ad esempio, un elettrone può emettere un fotone ($\gamma$) "virtuale", che a sua volta si scinde in una coppia elettrone-positrone ($e^-, e^+$) virtuale. Questa coppia vive per un istante brevissimo, per poi annichilarsi di nuovo in un fotone, che viene infine riassorbito dall'elettrone originale. Questo processo è chiamato "vestizione virtuale" (\textit{virtual dressing}) dell'elettrone.

Queste particelle "virtuali" non violano i principi di conservazione globalmente, perché esistono solo all'interno dei limiti imposti dall'indeterminazione energia-tempo. Tuttavia, se una sonda esterna (ad esempio, un fotone ad alta energia) interagisce con l'elettrone durante una di queste fluttuazioni, può fornire l'energia mancante (maggiore di $2m_ec^2$) per rendere "reali" le particelle della coppia virtuale, proiettandole nel mondo osservabile.

\subsection{Verso la Teoria Quantistica dei Campi}

La crisi della funzione d'onda e la necessità di descrivere la creazione e l'annichilazione di particelle ci spingono verso un nuovo formalismo: la \textbf{Teoria Quantistica dei Campi (QFT)}.

In QFT, le entità fondamentali non sono le particelle, ma i \textbf{campi}. Così come il fotone è l'eccitazione quantistica del campo elettromagnetico, l'elettrone sarà visto come l'eccitazione quantistica di un "campo elettronico".

Questi campi quantistici non hanno l'interpretazione probabilistica di Born della funzione d'onda. Matematicamente, sono \textbf{operatori} definiti in ogni punto dello spaziotempo, che possono essere espressi in termini di \textbf{operatori di creazione e annichilazione}, analoghi a quelli studiati per l'oscillatore armonico quantistico. Un operatore di creazione, agendo sul vuoto, "crea" una particella (un'eccitazione del campo), mentre un operatore di annichilazione la distrugge. Questo formalismo è intrinsecamente progettato per gestire un numero variabile di particelle.

\subsubsection{Antiparticelle e Causalità}

La QFT fornisce anche una spiegazione profonda dell'esistenza delle antiparticelle, legata alla struttura causale della relatività.
Consideriamo un processo in cui un elettrone viene emesso in un evento spaziotemporale $X$, interagisce (scattering) in un evento $Z$ e viene assorbito in un evento $Y$.

Se gli eventi $X$ e $Z$ sono separati da un intervallo di tipo \textbf{spazio} (\textit{space-like}), l'ordine temporale tra di essi non è assoluto: dipende dall'osservatore inerziale.
\begin{itemize}
    \item Un osservatore $O$ potrebbe vedere l'elettrone emesso in $X$ (al tempo $t_X$) e poi interagire in $Z$ (al tempo $t_Z > t_X$).
    \item Un altro osservatore $O'$ in moto relativo potrebbe vedere l'interazione in $Z$ (al tempo $t'_Z$) avvenire \textit{prima} dell'emissione in $X$ (al tempo $t'_X > t'_Z$).
\end{itemize}

La teoria deve fornire una descrizione coerente per entrambi. La QFT risolve questo paradosso in modo elegante: l'osservatore $O'$ non interpreta il processo come un elettrone che viaggia da $X$ a $Z$ "indietro nel tempo", ma come una \textbf{antiparticella (positrone)} che viaggia \textit{avanti nel tempo} da $Z$ a $X$.

Il fatto che le ampiezze di probabilità per la propagazione tra punti separati da intervalli di tipo spazio siano diverse da zero nella teoria relativistica, implica necessariamente l'esistenza delle antiparticelle.


\subsection{Conclusione e Prospettive}

Questa prima lezione ha mostrato come il tentativo di unire Meccanica Quantistica e Relatività Speciale non sia un esercizio banale. Ci costringe a un cambio di paradigma:
\begin{enumerate}
    \item Abbandonare il concetto di funzione d'onda per una singola particella.
    \item Accettare che il numero di particelle non è conservato ad alte energie.
    \item Introdurre le antiparticelle come una necessità fondamentale della teoria.
    \item Adottare il formalismo dei \textbf{campi quantistici} come descrizione fondamentale della materia.
\end{enumerate}
Nelle prossime lezioni, svilupperemo gli strumenti matematici di questo nuovo formalismo, imparando a calcolare le probabilità di processi come la creazione di coppie e lo scattering di particelle, entrando nel dominio della \textbf{Elettrodinamica Quantistica (QED)}.

